\section{Conclusion} % and Future Work
In this paper, we have introduced the problem of estimating species trees from a set of gene trees as an MOP. We have shown examples where the existing method, optimizing a single criterion, can overshoot the criterion and thus deviate from the true species tree. We have selected three objectives from three existing methods. Unlike traditional MOPs, optimizing an objective beyond a certain limit and discarding a dominated at an early generation can reduce the chance of generating better trees. Hence we cannot expect PF, sampled by a classical EMO algorithm, to contain highly accurate trees. Therefore, we have designed a specialized EMO algorithm, namely, SNOGA, which is a modification of NSGAII. %We have shown that SNOGA can generate a tree-space containing highly accurate trees. 
We have analyzed the behavioral difference between SNOGA and NSGAII on a collection of challenging simulated datasets and found that the best trees offered by SNOGA are much better than the best trees generated by NSGAII. Finally, we have shown that the tree-space generated by SNOGA contains highly accurate trees in comparison with widely used methods that optimize a single criterion. We are currently working to devise a systematic methodology to extract a limited number of relatively better trees from the final population without the knowledge of the true tree to effectively process biological datasets. Besides, we are modifying another popular EMO algorithm, namely, MOEA/D, to effectively solve this problem.

%We are currently working to improve SNOGA so that it can efficiently process large biological datasets. 

%Finally, we found the accuracy of the best trees offered by SNOGA is quite better three existing methods on a collection of challenging simulated datasets. 

%We are currently working to devise a systematic methodology to filter a limited number of better trees from the final population without the knowledge of the true tree provided with the simulated dataset. At present, our mutation selects one from NNI/SPR/TBR at random with equal probability. As a future improvement, we will improve it by adaptively adjusting the selection probabilities based on the success rate of an operator in the previous generation. Also, we are planning to emded domain knowledge inside NNI, SPR and TBR so that they can make informed (as opposed to random) rearrangement in the given tree. To enable SNOGA processing large datasets within a reasonable time, we will improve the efficiency of objective evaluations. Moreover, we are modifying another popular EMO algorithm, namely, MOEA/D, to effectively solve this problem.