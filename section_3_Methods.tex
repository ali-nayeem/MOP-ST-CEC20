\section{Methodology}
\label{sec:method}
In this paper, we adapt NSGAII for species tree estimation by integrating problem-specific encoding, initialization, crossover and mutation. Moreover, we design a special purpose EMO algorithm by modifying NSGAII considering the issues mentioned in Section~\ref{sec:problem}. In this section, we discuss the design of our modified EMO algorithm (pseudo-code shown in Algorithm~\ref{alg:nossga}) along with its different components. In what follows, the following definition will be useful. %associated with the problem 

\begin{definition}\label{def:domination_threshold}
	\small
	We call the worst value of a particular objective in the obtained PF of a certain generation as the \textbf{domination threshold} for that objective at that generation.
\end{definition}

\subsection{Algorithm Design}
One idea to resolve issue~\ref{item:i1}, could be to enforce the EMO algorithm being applied to maintain a population that contains some solutions having objective values slightly below the domination threshold (Definition~\ref{def:domination_threshold}). However, any such algorithm, in general, would continue optimizing one objective to a great extent even at the loss of some other objectives (resulting in issue~\ref{item:i1}) which may be detected through a change in sign of correlation coefficient (positive to negative)  between a pair of objectives as the generation changes. 
Therefore, we modify NSGAII to tackle issue~\ref{item:i1} in a better way than the original NSGAII in addition to resolving issue~\ref{item:i2}.  

\begin{equation}\label{eqn:nos}
\small
SNO(x) = \sum_{i=1}^{3} \frac{F_i(x)-z_i^{min}}{z_i^{max}-z_i^{min}}
\end{equation}
Here $x$ is a candidate solution (i.e., species tree) and $z_i^{min}$ ($z_i^{max}$) is the minimum (maximum) value of $i^{th}$ objective $F_i$ observed so far during the search process.

To overcome issue~\ref{item:i2}, we avoid the concept of non-domination while selecting solutions to form the population for the next generation (line~\ref{line:next_pop1}-\ref{line:next_pop2} of Algorithm~\ref{alg:nossga}). Instead, we sort solutions (line~\ref{line:sort_nos} of Algorithm~\ref{alg:nossga}) based on the summation of normalized objective (SNO) values as defined by the $SNO()$ function in Equation~\ref{eqn:nos}. That is why, we refer to our algorithm as SNOGA, i.e, Summation of Normalized Objectives based Genetic Algorithm. We are inspired by~\cite{qu2010multi} where a similar strategy was utilized to design an EMO algorithm. Moreover, as an attempt to address issue~\ref{item:i1}, we will not allow the selection of individuals that may arise conflict (negative correlation coefficient) between any pair of objectives. We accomplish this by using the tournament selection~\cite{goldberg1991comparative} based on $SNO()$ (line~\ref{line:tournament_nos}) while forming the offspring population.
%instead of the default binary tournament. 
Because we expect that selection pressure based on $SNO()$ will prefer those solutions that achieve reasonably well values across all objective than those solutions which are optimized along one objective heavily at the cost of other objectives. This feature might help us to resolve or at least lessen issue~\ref{item:i1} compared to the original NSGAII algorithm (this is supported by results presented in Section~\ref{sec:experiment}). %at least better than NSGAII. %as well as address issue~\ref{item:i2} that NSGAII cannot handle. 

Additionally, we always randomly choose one from among multiple solutions having the same $SNO()$ value (line \ref{line:diversity}) to help maintain the diversity of solutions. Also, we attempt to preserve diversity by using the crowding distance concept of NSGAII, albeit in a different way than NSGAII, by keeping some sparsely spaced (in the objective space) solutions having poor $SNO()$ values (line~\ref{line:crowding_s}-\ref{line:crowding_e}) in the population. In addition to doing tournament selection (as mentioned above), unlike NSGAII, we also use the random selection (line~\ref{line:random_selection}) to ensuring the participation of a few poor solutions (in terms of $SNO()$) in the offspring generating process. This increases the exploration capability of SNOGA to help escape local optima. Moreover, we expect this feature to aid in addressing issue~\ref{item:i2} as well.


%It optimizes three objectives, with certain attention to the nature of the problem, and generates a tree-space that contains high-quality species trees. 
%We also apply NSGAII, using the same crossover, mutation and initialization method as the SNOGA, to approximate the PF by optimizing the three objectives. 

\begin{algorithm}[!htbp]
	\scriptsize
	\caption{SNOGA}
	\textbf{Input:} $m_g$ (max. generations), $p_s$ (population size), $c_r$ (crossover rate), $m_r$ (mutation rate), $t_s$ (tournament size)\\
	\textbf{Output:} $P$ (A list of $p_s$ candidate solutions i.e., species trees)
	\begin{algorithmic}[1]\label{alg:nossga}
		\STATE{$P \gets$ population\_initialization($N$)} \COMMENT{Sec.~\ref{subsec:init}}
		\STATE{For each solution in $ P $, evaluate the objective functions and $SNO()$ function} \COMMENT{using Eqn.\ref{eqn:nos}}
		%\STATE{For each solution $x$ in $ P $, calculate SNO($x$)}
		\STATE{$g \gets 0$} \COMMENT{generation counter}
		\WHILE{ $g < m_g$}
		\STATE{$Q \gets \emptyset$} \COMMENT{offspring population, an empty list that can hold $p_s$ solutions}
		\FOR{$i \leftarrow 1$ to $p_s$}    %\label{mainLoop}    
		\STATE{$S_1 \gets$ tournament\_selection($P$, $t_s$)} \COMMENT{based on value of $SNO()$ function} \label{line:tournament_nos}
		\STATE{$S_2 \gets$ random\_selection($P$)} \label{line:random_selection}
		\STATE{$x \gets$ mutation(crossover($S_1, S_2, c_r$), $m_r$)}\COMMENT{Sec.~\ref{subsec:crossver}, \ref{subsec:mutation}}
		\STATE{$\!\!$Add $x$ to tail($Q$)}
		%\STATE{Generate an offspring by applying crossover on $S_1, S_2$, then mutate the offspring and append the resultant solution to $Q$}        
		\ENDFOR    
		\STATE{For each solution in $ Q $, evaluate the objective functions and $SNO()$ function }
		\STATE{$ R \gets P \cup Q$} \COMMENT{$R$ is a list holding $2p_s$ solutions}
		%\STATE{For each solution $x$ in $ R $, calculate SNO($x$)}
		\STATE{Sort the members of $ R $ in ascending order of $SNO()$ values} \COMMENT{ascending because all objectives are treated as minimization}\label{line:sort_nos} \label{line:nos}
		\STATE{$P \gets \emptyset$, Remove the solution from head($R$) and add it to tail($P$)} \label{line:next_pop1}
		%\STATE{$P[1] \gets R[1]$}
		%\STATE{$i \gets 2$}
		\FOR{$i \leftarrow 2$ to $p_s$}    %\label{mainLoop}  
		\STATE{Remove the solution from head($R$) and store it in $x$}    %\COMMENT{each element is shifted left by 1 position }   
		\STATE{\textbf{if} $SNO$($x$) $\ne$ $SNO$(solution at tail($P$)), \textbf{then} Add $x$ to tail($P$) }    \label{line:diversity}
		
		\ENDFOR    \label{line:next_pop2}
		%            \FOR{$i \gets 1$ to $N$}    %\label{mainLoop}    
		%                \IF{ SNO($R[i]) \ge P[i-1]$)}
		%                    \STATE{$P[i] \gets R[i]$}
		%                \ENDIF
		%                \STATE{Remove $R[i]$ from $R$}        
		%            \ENDFOR    
		%\STATE{For each remaining solution in $ R $, calculate the crowding distance}\label{line:crowding_s}
		\STATE{Sort the members of $ R $  in descending order of their crowding distance~\cite{deb2002fast} in $R$} \label{line:crowding_s}
		\STATE{Fill-up the remaining solutions for $P$ from the top of $R$}\label{line:crowding_e}
		\STATE{$g \gets g + 1$}
		\ENDWHILE
		\STATE{\textbf{return} $P$}
	\end{algorithmic}
\end{algorithm}

\subsection{Crossover}\label{subsec:crossver}
We use Prune-Delete-Graft (PDG)~\cite{villalobos2018memetic} as the crossover operator which generates one tree (i.e., offspring) from two parents. At first, it takes a random sub-tree from one of the parents. Then, it inserts the selected sub-tree in the other parent at a
randomly selected insertion point. Finally, it deletes duplicated species from the second tree and returns it as the output.

\subsection{Mutation} \label{subsec:mutation}
Our mutation operator applies one of the three widely used tree rearrangement strategies~\cite{felsenstein2004inferring}: (i) Nearest Neighbour
Interchange (NNI), (ii) Sub-tree Pruning and Re-grafting (SPR) and (iii) Tree Bisection and Reconnection (TBR). Each of them is selected randomly with equal probability. NNI exchanges sub-trees from an arbitrary internal branch to obtain a new tree. On the other hand, SPR picks a random sub-tree from a tree, removes it and then re-grafts it in a random position to generate a new tree. And TBR is a combination of SPR and NNI.

\subsection{Population Initialization}\label{subsec:init}
We generate the required number of solutions for the initial population by utilizing the given set of gene trees. To produce a single solution, we randomly pair two gene trees and apply our crossover operator on them with probability 1.0. 


\subsection{Implementation Notes}
We encode species tree using \textit{TreeTemplate} class provided by BIO++~\cite{gueguen2013bpp} which is a collection of C++ libraries for Bioinformatics. To implement NSGAII and SNOGA, we use a C++ framework for EMO, jMetalCpp~\cite{lopez2013jmetalcpp}. We reuse the implementation of PDG, NNI, SPR and TBR from~\cite{zambrano2016mo} after fixing some minor bugs therein. Furthermore, we evaluate each objective (QT/TP/PL) using a feature provided by each method (ASTRAL/STELAR/MP-EST) to score an existing species tree. Thus for each candidate solution, we invoke the executable of a particular method to evaluate an objective.
%\footnote{\url{https://github.com/jMetal/jMetalCpp}}
\begin{comment}
\subsection{Objective Evaluation}
Each method (ASTRAL, STELAR or MP-EST) provides an option to score an existing species tree. We used this feature to enable to evaluate the objectives. Thus for each candidate species tree, we invoke the executable of a particular method (ASTRAL/STELAR/MP-EST) to evaluate an objective. %We converted each objective into minimization to conform 
%We evaluated each objective by executing the method that 
%: invoke java executable
%MP-EST: Maximize branch length, c++ executable
\end{comment}